\chapter{Experimentos e Resultados}\label{cap-experimentos}

\section{Base de Dados}

%A base de dados � um elemento essencial para o procedimento de experimentos deste sistema de detec��o de pontos fiduciais pois disponibiliza as imagens utilizadas no procedimento de treino e teste. A base de dados auxilia o sistema de detec��o de pontos fiduciais nas caracter�sticas para o treino do algoritmo SVM e ao final para a extra��o de caracter�sticas para no c�lculo do desempenho do sistema de detec��o de pontos fiduciais. Assim, de acordo com a base de dados, podemos identificar se o sistema de pontos fiduciais desenvolvido � robusto.

Nesta disserta��o, iremos utilizar duas bases de dados denominadas por \emph{BioID} e \emph{Feret}. A \emph{BioID} � uma base de dados caracterizada por apresentar varia��es da pose, condi��es de ilumina��o natural e  \emph{backgrounds} distintos. Estas caracter�sticas tornaram a \emph{BioID} uma base de dados amplamente utilizada em pesquisas para a compara��o de desempenho entre t�cnicas de classifica��o aplicadas � sistemas de detec��o de pontos fiduciais. A base de dados \emph{BioID} original � composta por $1521$ imagens em escala de cinza com resolu��o de $384 \times 286$. A \emph{BioID} possui imagens que foram retiradas de $23$ indiv�duos em diferentes formas. Neste conjunto, os indiv�duos podem conter �culos, barba e bigode. Al�m das imagens, a base de dados possui uma anota��o de $20$ pontos na face e outra contendo somente os olhos. Na Figura \ref{fig20}, podemos visualizar exemplos das imagens da base de dados \emph{BioID}.

%[width=8cm]
\begin{figure}[!htb]
  % Requires \usepackage{graphicx}
  \centering
  \includegraphics[scale=0.8]{bioid.eps}
  \caption[Exemplos de imagens da base dados \emph{BioID}.]{Exemplos de imagens da base dados \emph{BioID} \cite{bioID}.}
  \label{fig20}
\end{figure}

